\documentclass[../main.tex]{subfiles}

\begin{document}
    In diesem Kurs werden die ethischen und philosophischen Fragen rund um die Informatik besprochen. Außerdem nimmt die Auswirkung der Informatik auf die Gesellschaft und andersherum eine wichtige Rolle ein.
    
    \begin{quote}
        \emph{Dabei geht es nicht darum die Fragen von heute zu klären sondern in der Lage zu sein die Fragen von morgen zu lösen.}
    \end{quote}
    \clearpage

    \section{Programmierung und Software Engineering}
        Für den Unterschied zwischen Programmierung und Software Engineering spielt die Geschichte der Programmierung eine entscheidende Role.
        
        \subsection{Geschichte der Programmierung}
        
            \paragraph{40er und 50er Jahren}
                In den 40er und 50er Jahren war die Hardware teuer und hatte keine Betriebssysteme, sodass Computer nur von Experten für den Eigengebrauch programmiert wurden. Die so erstellten Programme hatten meistens nur genau eine Aufgabe und wurden danach nicht weiter verwendet.
                
            \paragraph{Ab den 60er Jahren}
                In den 60er Jahren gab es eine Wende da Hardware bezahlbar wurde, sodass auch normale Anwender Programme brauchten. Software hat sich unüberschaubar ausgeweitet und die Lebensdauer von Software war nun länger als von der Hardware.
            
        \subsection{Die Softwarekrise}
            Aufgrund des abrupten Wandels von Software, die nur von den Entwicklern bedient wurde, zu Software, die für jeden war, und die damit verbundenen Probleme ist es zu der sogenannten Software Krise gekommen. Die NATO SE-Konferenz prägte diesen Begriff 1968 mit folgenden Eigenschaften
            
            \begin{itemize}
                \item Software erfüllt nicht die Wünsche der Kunden.
                \item Software ist nicht zuverlässig.
                \item Softwareentwicklung und -wartung ist zu Zeit und Kostenintensive.
                \item Die Kosten für Software übersteigt die Kosten für Hardware.
                \item Nur 34\% der Softwareprojekte wurden erfolgreich abgeschlossen.
            \end{itemize}
            
            \subsubsection{Ursachen}
                \begin{itemize}
                    \item Software ist ein immaterielles Gut und hat damit keinen Materialwert.
                    \item Software ist schwer verständlich, sodass es zu Kommunikationsproblemen zwischen Auftraggeber und Entwickler kommt.
                    \item Software ist nur in den Wirkungen beim Ablauf auf Computern beobachtbar.
                    \item Software verschleißt nicht, sodass nur Erweiterungen angebaut werden. Dennoch kann Software altern, wie es beim 2000 Problem sichtbar geworden ist.
                    \item Software Fehler sind schwer zu erkennen.
                    \item Software ist scheinbar leicht zu verändern.
                    \item Kleinste Änderungen in einer Software können massive Änderungen im Verhalten der Software zur folge haben.
                    \item Nachweiß der wunschgemäßen Verhaltens ist schwierig.
                \end{itemize}
                
            \subsection{Schlussfolgerung}
                Aufgrund der Softwarekrise soll die Erstellung von Software nicht mehr eine kreative Kunst bleiben sondern eine Ingenieurwissenschaft mit wohldefinierten Vorgehensweisen werden. Deshalb wurde der Begriff des Software Engineering geprägt.
                
        \subsection{Software Engineering}
            Unter Software Engineering versteht man
            
            \begin{itemize}
                \item die Entwicklung,
                \item die Pflege und
                \item den Einsatz
            \end{itemize}
            von \emph{qualitativ hochwertiger} Software unter Einsatz von
            
            \begin{itemize}
                \item wissenschaftlichen Methoden,
                \item wirtschaftlichen Prinzipien,
                \item geplanten Vorgehensmodellen,
                \item Werkzeugen und
                \item quantifizierbaren Zielen.
            \end{itemize}
            
        \subsection{Usability Engineering}
            Eine Erweiterung des Software Engineering ist das Usability Engineering. Mit dem Aufkommen des \emph{Apple Lisa} in den 80ern merkte man, dass interaktive, technische System nicht unabhängig von der Arbeitswelt betrachtet werden können, sondern die Aufgaben und das Umfeld entscheidend für die Gestaltung des Systems ist. Seit den 90er Jahren ist es deshalb ein strategisches Ziel aller Unternehmen eine intuitive Benutzerschnittstelle zu haben.
                
            Verschärft wird der Druck durch das Konzept von \emph{Software as a service}, da die Konkurrenz nur ein Klick entfernt ist und ein Produkt auf den ersten Blick überzeugen muss.
\end{document}
