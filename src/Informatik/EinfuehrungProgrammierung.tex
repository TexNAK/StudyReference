\documentclass[../main.tex]{subfiles}

\begin{document}
    In diesem Kurs werden die Grundlagen der Programmierung mithilfe der funktionalen Programmiersprache \href{https://racket-lang.org}{Racket} vermittelt.
    \clearpage
    
    \section{Sprachen}
        \subsection{Maschinenebene}
            \paragraph{Maschinensprache}
                Die vollständige Liste mit Befehlen, die ein Computer versteht ist die Maschinensprache des Computers. Jeder Rechner-Architektur hat eine andere Maschinensprache.
                
            \paragraph{Maschinencode}
                Maschinencode ist eine ausführbare Darstellung eines Programms mit Maschinencode als Bitmuster. Maschinencode ist maschinenabhängig.
            
        \subsection{Assemblerebene}
            \paragraph{Assemblersprache}
                Assemblersprache ist eine textorientierte Darstellung der Maschinensprache zur besseren Lesbarkeit für den Menschen.
                
                Die Liste an Befehlen bleibt aber die selbe wie in der zu Grunde liegenden Maschinensprache. Zahlen können bereits als Dezimalzahlen geschrieben werden.
                
            \paragraph{Assemblercode}
                Assemblercode ist ein Programm in Assemblersprache.
        
        \subsection{Hochsprache}
            \paragraph{Hochsprache}
                Hohe Programmiersprachen sind formale Sprachen zur textuellen Darstellung von Programmen. Außerdem versuchen Programmiersprachen die Konzepte der von-Neumann Architektur zu verstecken und zu abstrahieren.
                % TODO Add reference to von-Neumann Architektur
                
            \paragraph{Quellprogramm}
                Quellprogramm\footnote{Auch: Sourcecode} ist ein Programm in einer Hochsprache.
\end{document}
