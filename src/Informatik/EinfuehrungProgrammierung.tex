\documentclass[../main.tex]{subfiles}

\begin{document}
    In diesem Kurs werden die Grundlagen der Programmierung mithilfe der funktionalen Programmiersprache \href{https://racket-lang.org}{Racket} vermittelt.
    \clearpage
    
    \section{Sprachen}
        \subsection{Maschinenebene}
            \paragraph{Maschinensprache}
                Die vollständige Liste mit Befehlen, die ein Computer versteht ist die Maschinensprache des Computers. Jeder Rechner-Architektur hat eine andere Maschinensprache.
                
            \paragraph{Maschinencode}
                Maschinencode ist eine ausführbare Darstellung eines Programms mit Maschinencode als Bitmuster. Maschinencode ist maschinenabhängig.
            
        \subsection{Assemblerebene}
            \paragraph{Assemblersprache}
                Assemblersprache ist eine textorientierte Darstellung der Maschinensprache zur besseren Lesbarkeit für den Menschen.
                
                Die Liste an Befehlen bleibt aber die selbe wie in der zu Grunde liegenden Maschinensprache. Zahlen können bereits als Dezimalzahlen geschrieben werden.
                
            \paragraph{Assemblercode}
                Assemblercode ist ein Programm in Assemblersprache.
        
        \subsection{Hochsprache}
            \paragraph{Hochsprache}
                Hohe Programmiersprachen sind formale Sprachen zur textuellen Darstellung von Programmen. Außerdem versuchen Programmiersprachen die Konzepte der von-Neumann Architektur zu verstecken und zu abstrahieren.
                % TODO Add reference to von-Neumann Architektur
                
            \paragraph{Quellprogramm}
                Quellprogramm\footnote{Auch: Sourcecode} ist ein Programm in einer Hochsprache.
                
    \section{Auswertungsregeln}
        \paragraph{Syntax}
            Die Syntax einer Programmiersprache legt zulässige Elemente und Struktur fest.
            
        \paragraph{Semantik}
            Die Semantik einer Programmiersprache legt die Bedeutung von syntaktisch korrekten Ausdrücken fest. Die Semantik funktionaler Programmiersprachen orientiert sich an der Semantik der Mathematik.
            
        \paragraph{Ersetzungen}
            Um die Bedeutung eines Ausdrucks zu bestimmen, werden nacheinander semantisch äquivalente Umformungen mit dem \hyperref[section:Programmierung:Auswertungsregeln:ErsetzungsmodellFunktionsanwendungen]{\emph{Ersetzungsmodell für Funktionsanwendungen}} durchgeführt. Die Reihenfolge der Auswertung von Teilausdrücken in funktionalen Sprachen hat dabei keine Auswirkung auf das Endergebnis.
            
            Beim Ersetzen von Variablen mit Konstanten wird dieser Prozess als Auswertung bezeichnet. Die Reihenfolge der Auswertungen spielt dabei in funktionalen Programmiersprachen keine Rolle.
            
        \paragraph{Ersetzungsmodell für Funktionsanwendungen}
		\label{section:Programmierung:Auswertungsregeln:ErsetzungsmodellFunktionsanwendungen}
			Eine Funktion wird ausgewertet, in dem jeder formale Parameter im Funktionsrumpf durch das korrespondierende ausgewertete Argument ersetzt wird.
            
        \paragraph{Self Evaluating Expressions}
            \emph{Self Evaluating Expressions} sind Ausdrücke, die zu sich selbst ausgewertet werden. Die Auswertung eines Ausdrucks ist beendet, wenn eine solche Expressions erreicht wurde. Beispiele sind Zahlen sowie die Wahrheitswerte $\#t$ und $\#f$.
            
        \paragraph{Funktionsaufruf}
            Bei strikter Auswertung werden erst alle Argumente eines Funktionsaufrufs berechnet. Dieses Verfahren kommt bei Racket zum Einsatz. Diesem Verfahren steht die lazy Auswertung gegenüber, welche zum Beispiel in Haskel verwendet wird, bei dem auch die Argumente erst berechnet werden, wenn sie benötigt werden.   
            
    \section{Aufschreibregeln für Funktionen}
        \paragraph{Regel 1}
            Verwende aussagekräftige Namen für Funktionen und deren Parameter.
        
        \paragraph{Regel 2}
            Beschreibe den Zweck der Funktion in einem Satz mithilfe eines Kommentars in der Zeile vor der Funktionsdefinition. Der Satz sollte das Resultat der Funktion in Abhängigkeit von den Parametern beschreiben.
            
        \paragraph{Regel 3}
            Füge unter der Funktion Tests hinzu, die die Richtigkeit der Funktion auch in Zukunft sicherstellen. Dazu müssen Test für alle Randbereiche vorhanden sein.
            
        \paragraph{Regel 4}
            Definiere für jeden Zusammenhang zwischen Größen, die sich aus der Problemstellung ergeben, eine Funktion.
            
        \paragraph{Regel 5}
            Ersetze jede Konstante, deren Bedeutung sich nicht aus dem Kontext ergibt, durch einen sprechenden Variablennamen.
            
    \section{Datenabstraktion}
        \subsection{Abstraktionsbarrieren}
            Beispiel 
            
            \begin{itemize}
                \item \textbf{Punkt in der Sicht der Anwendungsprogramme}
                \item ... add-point equal-point? ...
                \item \textbf{Punkte bestehen aus x- und y-Koordinaten}
                \item ... make-point point-x point-y
                \item \textbf{Punkt implementiert als pairs}
                \item ... cons car cdr ...
                \item \textbf{Interne Implementation von pairs}
            \end{itemize}
        

\end{document}
