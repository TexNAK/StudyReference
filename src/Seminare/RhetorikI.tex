\documentclass[../main.tex]{subfiles}

\begin{document}
    
    Hört man den Begriff Rhetorik so denkt ein mancher direkt an die Linguistik. Die große Familie der rhetorischen Mittel, was für eine Wirkung sie erzielen und wie man sie zielführend einsetzt. Doch entgegen diesem Meinungsbild deckt die Rhetorik noch viel mehr ab.
    \begin{quote}
        Rhetorik (altgriechisch): "Die Redekunst"
    \end{quote}
    Per Definition umfasst die Rhetorik nicht nur den Bereich der Linguistik sondern darüber hinaus ebenfalls Bereich wie die Körpersprache mit Gestik und Mimik sowie die Proxemik also das Raumverhalten. In den folgenden Sektionen werden unter anderem diese Themen im Kontext des Seminar Rhetorik I behandelt.
    
    \clearpage
    
    \section{Vorbereitung}
        Bei einem Vortrag jeglicher Art ist eine gute Vorbereitung essentiell. Sie bietet einem Sicherheit bei der Präsentation der Inhalte und potentielle Auswege aus unangenehmen Situationen. Die Vorbereitung betrifft mehrere Bereiche auf die in den folgenden Abschnitten näher eingegangen wird:
        \begin{enumerate}
            \item Gesprochenes
            \begin{enumerate}
                \item Publikumsbezug
                \item Wirkungskurve
            \end{enumerate}
            \item Medien
            \begin{enumerate}
                \item Foliensatz
                \item Flipchart
                \item Metaplanner\footnote{Im allgemeinen Sprachgebrauch auch als Pinnwand bezeichnet.}
                \item Smart-/Whiteboard
            \end{enumerate}
        \end{enumerate}
        
        \subsection[Zielsetzung]{Zielsetzung und Selbstbewusstsein}
            Viele tendieren dazu einen perfekten Vortrag anzustreben in dem alle zuvor vorbereiteten Aspekte souverän präsentiert und verständlich dargestellt werden. Ironischerweise resultiert eben genau diese Zielsetzung darin, dass die Rede alles andere als geplant verläuft und eben dieses Streben nach Perfektion das Gegenteil bewirkt. Empfehlenswert ist es eine niedrige Erwartungshaltung einzunehmen und zum Beispiel die Vorbereitung dahingehend zu strukturieren welche Punkte signifikant sind und welche bei Bedarf übersprungen werden können. Mehr dazu im Abschnitt \ref{section:Seminar:RhetorikI:Wirkungskurve}.
            \begin{quote}
                \emph{Fehler sind natürlich und menschlich. Niemand wird einen vierteilen, weil man sich verhaspelt, verspricht oder den roten Faden verliert.}
            \end{quote}
    
    \section{Körpersprache}
        \subsection{Erster Eindruck}
            Der Erste Eindruck. Bei den meisten dauert es unterbewusst nur wenige Millisekunden bis sie sich das erste Bild von einer bisher unbekannten Person gebildet haben. In den folgenden Sekunden bildet sich das Bewusstsein ein genaueres Bild von der Person. Bis man das erste Wort gesprochen hat ist die Zeit des ersten Eindrucks also schon längst vergangen.\\
            In den folgenden Sektionen werden die wichtigsten Aspekte erläutert, die auf das Publikum noch vor Beginn der Präsentation beeinflussen.
            
            \subsubsection{Sitzposition}
                Es gibt unzählige Sitzpositionen. Einige sind gemütlich, andere weniger. Aspekte wie verschränkte Arme oder überschlagene Beine wirken eher abweisend, verschlossen oder auch nachdenklich wohingegen ein fester, zentrierter Stand der Füße sowie eine aufrechte, aber nicht versteifte Position Professionalität und Kompetenz ausstrahlen. Man beachte, dass nicht jede Haltung jeder Situation angemessen ist. Der beste Weg um herauszufinden, wie bestimmte Sitzpositionen wirken ist andere Leute zu beobachten. Wie sitzen sie? Wie wirkt ihre Position auf einen?\\
                \\
                Nun steht man auf um die Präsentation zu beginnen. Hier kommt es häufig vor, dass unterbewusst die Hände zur Hilfe genommen werden, was im besten Fall vermieden werden sollte. Sobald man aufgerichtet ist entsteht der natürliche Reflex die Kleidung zurecht zu rücken. Dies wirkt unprofessionell und in den allermeisten Fällen sitzt diese bereits angemessen.
                
            \subsubsection{erster Blickkontakt}
                Der unterbewusste, erste Eindruck ist geschaffen und das Publikum bildet sich eine bewusste, zweite Meinung über den Präsentierenden. An dieser Stelle kommt Blickkontakt ins Spiel. Nicht sofort Richtung Bühne los sprinten sondern einen Moment verweilen. Ruhe und Gelassenheit ausstrahlen und diese wertvollen Momente benutzen um einen Blick in die Runde zu wagen. Einen schnellen Blick über die Zuhörerschaft schweifen lassen, kurzzeitig mit den Augen auf bestimmten Personen verweilen und dann die Reise zur Präsentationsfläche antreten.
                
                \paragraph{Aufgang} Der Weg zur Bühne sollte auf keinen Fall dafür verwendet werden den Boden genauer zu inspizieren sondern stattdessen dazu verwendet werden einen langsamen Blick in die Runde zu werfen und sich Sympathie-Inseln zu suchen (dazu mehr in Abschnitt \ref{section:Seminar:RhetorikI:SympathieInsel}), während man idealerweise und sofern möglich in einem Halbkreis auf die Bühne schreitet.
        
        \subsection{Blickkontakt}
        	\label{section:Seminar:RhetorikI:SympathieInsel}
            Während einer Präsentation ist es essentiell eine Verbindung zu dem Publikum aufzubauen anstatt stumpf seine Informationen wie ein Wasserfall zu verteilen und anschließend die Bühne wieder zu verlassen. Ein wichtiger Aspekt bei der Sympathiebildung ist stetiger Blickkontakt. Nun bedeutet Blickkontakt nicht, dass man sich eine Person im Publikum, womöglich noch den wichtigsten Rezipienten und Entscheidungsträger, herauspickt und diese über die Dauer des Vortrags fixiert. 
            \begin{quote}
                \emph{Man sollte sich immer wieder wechselnde Personen suchen, denen man zwischen einer und drei Sekunden\footnote{Dies ist ein pauschaler, statistischer Wert und kann je nach Person weit darüber oder darunter liegen!} in die Augen schaut.}
            \end{quote}
            Dabei entsteht im Idealfall eine Sympathie bei dem Empfänger und dieser fühlt sich wahrgenommen. Meist ist dies merkbar durch ein leichtes Lächeln oder aber ein unterbewusstes Blinzeln. Dies signalisiert meist auch den Punkt an dem man die Person wechseln sollte bevor die Sympathie in ein Unwohlsein umschwenkt und sich die Person beobachtet fühlt. In der Fachliteratur wird ein solcher Blickkontakt meist als \emph{Sympathie-Insel} bezeichnet.
            
        \subsection{Gestik}
    		% TODO Ausformulieren
        	\begin{itemize}
        		\item Füße schulterbreit
        		\item Hände nicht verschränken
        		\item Offene Gesten, nicht den Bauch verdecken
        	\end{itemize}
            
            
    
    \section{Auftritt}
        \subsection[Pausen]{Ähm, Ehh, Ja - Die Wirkung von Pausen}
            Um den Informationsdurst der gierigen Zuhörer zu stillen tendieren wir dazu einen konstanten Redefluss anzustreben. Nun gibt es jedoch wenige Menschen die in der Lage sind kontinuierlich zu reden und gleichzeitig darüber nachzudenken was sie als nächstes sagen werden.\\
            Damit der Durst der Zuhörer gestillt wird tendieren viele dazu Laute wie z.B. \emph{ähm} oder auch \emph{ehh} von sich zu geben um die Rezipienten nicht auf die nächsten Informationen warten zu lassen. Solche Laute lassen den Vortragenden unsicher und unvorbereitet wirken, was absolut unerwünscht ist.\\
            Nun kann man sich vornehmen gezielt auf solche Laute zu achten und diese im gleichen Zuge vermeiden. Effektiv funktioniert diese Technik in den wenigsten Fällen.
            \begin{quote}
                \emph{Bedachte Pausen geben Bedenkzeit und können eine starke Wirkung erzeugen.}
            \end{quote}
            Während einer Präsentation ist man selbst die Regie. Braucht man eine kurze Bedenkzeit um den nächsten Aspekt überzeugend darzustellen so darf man sich diese nehmen. Sollte es komisch vorkommen mitten im Redefluss eine Pause zu machen so trinkt etwas oder öffnet das Fenster. Es wirkt natürlich und nimmt einem Vortrag den steifen Charakter. Pausen bieten auch nicht nur euch Bedenkzeit sondern können auch eine starke Wirkung erzielen, indem sie dem Publikum Zeit geben über das zuvor gesagte nachzudenken und zu reflektieren.
            
        \subsection{Sprechgeschwindigkeit}
        	Viele tendieren dazu sehr schnell zu sprechen, wenn sie nervös sind oder sich mit einem Thema sehr gut auskennen. Während einer Präsentation kann nicht nur die Betonung genutzt werden um bestimmte Passagen hervorzuheben sondern auch die Geschwindigkeit mit der diese vorgetragen werden. Zum Beispiel kann ein Satz partiell langsamer gesprochen werden um einen besonderen Fokus auf diesen zu legen. Umgekehrt ist es auch möglich unwichtige Abschnitte schneller vorzutragen. Dabei sollte jedoch nie so schnell gesprochen werden, dass die Inhalte nichtmehr verständlich sind.\\
        	Meist ist es sehr schwierig selbst die richtige Sprechgeschwindigkeit zu finden. Es kann zum Beispiel helfen, wenn man eine bekannte Person im Publikum (zum Beispiel ein Freund o.ä.) bittet auf die Verständlichkeit zu achten und einen unauffällig darauf aufmerksam zu machen, dass man zu schnell spricht. 
        
        \subsection{Wirkungskurve}
        	\label{section:Seminar:RhetorikI:Wirkungskurve}
        	% TODO Ausformulieren
        	\begin{itemize}
        		\item Einleitung
        		\begin{itemize}
        			\item Catcher
        			\item Fragestellung
        			\item Witz
        			\item Zitat
        		\end{itemize}
        		\item Hauptteil
        		\begin{itemize}
        			\item Drei Hauptaspekte in der Reihenfolge 2, 1, 3 (1=schwächstes)
        			\item Zwischen Aspekten jeweils Nebeninformationen und sonstiges einstreuen.
        		\end{itemize}
        		\item Schluss
        		\begin{itemize}
        			\item Zusammenfassung der drei Hauptaspekte
        			\item Closer
        			\item Aufforderung/Aufruf
        			\item Zitat (potentiell auf Eingangszitat wiederholt eingehen)
        			\item NICHT "Danke", "Vielen Dank für ihre Aufmerksamkeit" sagen
        			\item Fragestellungen nicht im Schluss sondern idealerweise während des Hauptteils oder sofern zu extensiv nach der Präsentation individuell besprechen. 
        		\end{itemize}
        	\end{itemize}
            % \subsubsection{Hauptaspekte}
            
    \section{Proxemik}
    	% TODO Ausformulieren
    	\begin{itemize}
    		\item Medien am Rand, Redner in der Mitte
    		\item Klaren Abstand zu Medien (nicht hinter Flipchart "verstecken")
    		\item Partner im Halbkreis hinter einem
    		\item Bei Hauptpunkten fest in der Mitte stehen 
    		\item Bewegen! Schritte machen statt nur Gewicht zu verlagern.
    		\item Standbein/Spielbein
    	\end{itemize}
    
    \section{Medieneinsatz}
    	% TODO Ausformulieren
    	\begin{itemize}
    		\item Folien zeitweise ausschalten (Blank)
    		\item Drei Hauptaspekte (ref Wirkungskurve) -> evtl. nichtmal auf Folien
    	\end{itemize}
    
    \section*{Danksagung}
    	Zum Abschluss möchten die Autoren einen großen Dank an Jan Friedrichs und sein Team von Adventure Learning ausrichten, die sehr viel Zeit und Ehrgeiz investieren um den Studenten der Nordakademie die Kunst der Rhetorik nahezulegen.
        % TODO Thank Jan and the whole crew from Adventure Learning for the weekend and recommend it to the reader.
    
\end{document}