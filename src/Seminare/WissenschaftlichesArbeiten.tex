\documentclass[../main.tex]{subfiles}

\begin{document}
    In diesem Pflichtseminar wird das wissenschaftliche Arbeiten genauer betrachtet, dass eine Werkzeug für alle anderen Kurse darstellt.
    \clearpage

    \section{Wissenschaft}
        Damit etwas als Wissenschaft anerkannt wird, muss es mindestens folgende Bedingungen erfüllen:
        
        \begin{itemize}
            \item Alle Aussagen müssen begründet sein.
            \item Es muss eine zusammenhängende Menge von Aussagen sein.
            \item Die Aussagen müssen mit einer argumentativen Struktur verbunden sein.
        \end{itemize}
        
    \section{Wissenschaftliches Arbeiten}
    
        \paragraph{Vorgehen}
        \begin{enumerate}
            \item Sichten
            \item Lesen
            \item Denken
            \item Schreiben
        \end{enumerate}
        
    \section{Wissenschaftliche Forschung}
        Es gibt zwei verschiedene Forschungsperspektive: explorative und hypothesenüberprüfend. Der Ablauf ist aber unabhängig von der Forschungsperspektive immer gleich:
    
        \begin{enumerate}
            \item Forschungsgegenstand, Hypothesen und Forschungsfragen
            \item Operationalisieren
            \item Forschungsdesign
            \item Durchführung
        \end{enumerate}
        
    \section{Erhebungsmethoden}
        \subsection{Quantitative Erhebungsmethoden}
            \begin{itemize}
                \item Standardisierte Interviews
                \item Strukturierte Beobachtungen
                \item Befragungen
                \item Experimente
            \end{itemize}
        
        \subsection{Qualitative Erhebungsmethoden}
            \begin{itemize}
                \item Qualitative Interviews
                \item Gruppendiskussionen
                \item Dokumentenanalyse
                \item Beobachtungen
            \end{itemize}
    
\end{document}
