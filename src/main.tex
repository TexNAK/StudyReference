\documentclass[a4paper]{report}
% \documentclass[a4paper]{book}

% Imports
\usepackage{fontspec}
\usepackage[ngerman]{babel}
\usepackage{fancyhdr}
\usepackage{epigraph}
\usepackage{subfiles}
\usepackage[acronym,toc]{glossaries}
\usepackage{amsmath}
\usepackage{listings}
\usepackage{amssymb}
\usepackage{graphicx}
\usepackage{siunitx}
\usepackage{tkz-graph}
\usepackage[european]{circuitikz}
\usepackage{hyperref}


\usetikzlibrary{automata,arrows}

% Title page
\title{Angewandte Informatik \\
    \noindent\rule[0.25ex]{\linewidth}{0.5pt}
    \large Das einzig wahre Informatik-Studium!
}
\author{Til Blechschmidt}
\author{
  Blechschmidt, Til\\
  \texttt{til@blechschmidt.de}
  \and
  Peeters, Noah\\
  \texttt{noah.peeters@icloud.com}
}

% Numbering
\setcounter{secnumdepth}{3}
\setcounter{tocdepth}{2}

% Quote styling
\setlength\epigraphwidth{.8\textwidth}
\setlength\epigraphrule{0pt}


\begin{document}
    % Title page
    \thispagestyle{fancy}
    \maketitle

    % Abstract
     \begin{abstract}
         Das folgende Dokument fasst alle Inhalte zusammen, die die Autoren im Laufe des Studiengangs Angewandte Informatik sowie in zusätzlichen Seminaren an der Nordakademie vermittelt bekommen haben.
     \end{abstract}
     \newpage

    % TOC
    \tableofcontents

    % --- Vorlesungsinhalte ---
    \part{Diskrete Mathematik}
        \chapter{Boolesche Algebra}
            \subfile{DiskreteMathematik/BoolescheAlgebra}
        \chapter{Formale Grundlagen}
            \subfile{DiskreteMathematik/FormaleGrundlagen}

    \part{Informatik}
        \chapter{Kursübergreifenden Grundlagen}
            \subfile{Informatik/Grundlagen}
        \chapter{Technische Grundlagen}
            \subfile{Informatik/TechnischeGrundlagen}
        \chapter{Einführung in die Programmierung}
            \subfile{Informatik/EinfuehrungProgrammierung}
        \chapter{Informatik und Gesellschaft}
            \subfile{Informatik/InformatikGesellschaft}

    \part{Wirtschaft}
        \chapter{Allgemeine Betriebswirtschaftslehre}
            \subfile{Wirtschaft/ABWL}

    % --- Seminare ---
    \part{Seminarwesen}
        \chapter{Wissenschaftliches Arbeiten}
            \subfile{Seminare/WissenschaftlichesArbeiten}
        \chapter{Rhetorik I}
            \subfile{Seminare/RhetorikI}


    \clearpage
    \part{Appendix}

        \clearpage
        \printglossary[type=\acronymtype]
        \printglossary

        \clearpage
        \bibliographystyle{unsrt}
        \bibliography{references}

        \listoffigures

        \clearpage
        \chapter{Signifikante Meilensteine}

            \epigraph{\itshape Ich habe mich gerade daran gewöhnt, dass man Computer mit C schreibt ... Ich hasse Sprachen -.-}{---Noah Peeters}

            \epigraph{\itshape Das macht doch alles keinen Sinn mit C und K im Deutschen!}{---Noah Peeters}

    	    \epigraph{\itshape Wir sollten erst das Exposé fertig machen bevor wir googlen wie eine Toilette funktioniert.}{---Noah Peeters}
    	    
    	    \epigraph{\itshape Weißt du was uns das kostet, wenn wir einen Azubi bei uns in die Fußgängerzone stellen?!}

            \epigraph{\itshape Das epigraph Package ist perfekt!}{---Til Blechschmidt}

\end{document}
