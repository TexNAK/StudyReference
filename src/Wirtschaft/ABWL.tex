\documentclass[../main.tex]{subfiles}

\begin{document}
    In diesem Kurs geht es um Wirtschaftszeug.
    \clearpage

    \section{Definitionen}
        \subsection{Allgemein}
            \paragraph{Betrieb}
                Ein Betrieb ist ein orts- und zweckgebunden Einrichtung zur Erstellung von Gütern und Dienstleistungen.
                
            \paragraph{Firma}
                Ein Betrieb zur Erstellung von Gütern.
                
            \paragraph{Geschäftsbetrieb}
                Der Zweck des Unternehmens.
                
            \paragraph{Unternehmen}
                Ein Unternehmen oder eine Unternehmung ist eine örtlich nicht gebundene finanziell-wirtschaftliche und rechtliche Einheit.
                
            \paragraph{Gesellschaft}
                Eine Gesellschaft ist eine rechtliche Form.
            
            \paragraph{Holding}
                Ein Unternehmen zur Organisation mehrerer Tochterunternehmen.

        \subsection{Materialien in einem Unternehmen}
            \paragraph{Rohstoffe}
                In einer Tischlerei ist Holz der Rohstoff.
                
            \paragraph{Hilfsstoffe}
                In einer Tischlerei sind Leim, Nägel, Schrauben und Farbe Hilfsstoffe.
                
            \paragraph{Betriebsstoffe}
                In einer Tischlerei sind Strom, Wasser und Öl Betriebsstoffe.
                
            \paragraph{Betriebsmittel}
                In einer Tischlerei sind Maschinen Betriebsmittel.

        \subsection{Entlohnung}            
            \paragraph{Gehalt}
                Ein Gehalt ist ein feste, also immer gleiche, Entlohnung für Arbeitnehmer.
                
            \paragraph{Lohn}
                Lohn ist die variable Entlohnung für Hilfskräfte.
                
        \subsection{Eigentumsvorbehalt}
            \paragraph{Eigentumsvorbehalt}
                Eigentümer der Ware bleibt der Lieferant, bis die Ware bezahlt ist. Der Empfänger ist nur Besitzer der Ware.
            
            \paragraph{Verlängerter Eigentumsvorbehalt}
                Wenn die unter Eigentumsvorbehalt verkaufter Ware weiterverkauft wurde, dann kann der Lieferant das Geld von dem neuen Besitzer bekommen.
                
        \subsection{Kennzahlen}
            \begin{equation}
                \text{Ertrag} = \text{Output} \cdot \text{Verkaufspreis}
            \end{equation}

            \begin{equation}
                \text{Aufwand} = \text{Input} \cdot \text{Einkaufspreis}
            \end{equation}

            \begin{equation}
                \text{Gewinn} = \text{Ertrag} - \text{Aufwand}
            \end{equation}
            
            \begin{equation}
                \text{Produktivität} = \frac{\text{Output}}{\text{Input}}
            \end{equation}
            
            \begin{equation}
                \text{Wirtschaftlichkeit} = \frac{\text{Ertrag}}{\text{Aufwand}}
            \end{equation}
            
            \begin{equation}
                \text{Rentabilität} = \frac{\text{Erfolgsgröße}}{\text{Basisgröße}} \text{z.B.} \frac{\text{Gewinn}}{\text{Eigenkapital}}
            \end{equation}
            
    \section{Akteure in einem Unternehmen}    
        \begin{minipage}{12cm}
            \begin{tabular}{l l}
              Akteur & Ziele \\
              \hline
              Mitarbeiter & Leistungsgerechte Entlohnung, motivierende Arbeitsbedingungen etc. \\
              Kunden & billige Güter oder Dienstleistungen mit gewünschter Qualität \\
              Geschäftsführung & Gewinnmaximierung \\
              Management & Gehalt, Macht, Einfluss\\
              Eigenkapitalgeber\footnote{Eigenkapitalgeber können zum Beispiel Aktionäre sein.} & Renditen \\
              Fremdkapitalgeber\footnote{Fremdkapitalgeber können zum Beispiel Banken sein.} & Renditen\\
              Öffentlichkeit\footnote{Zur Öffentlichkeit gehört auch der Staat} & Steuern, Gesetzliche Regelungen, Umweltschutz etc.\\
              Wettbewerber & Höheren Marktanteil\\
              Lieferanten & Zuverlässige Bezahlung und langfristige Lieferantenbeziehung
            \end{tabular}
        \end{minipage}
        
        \paragraph{Shareholder und Stakeholder}
            Shareholder sind Eigenkapitalgeber, während alle anderen Akteure Stakeholder sind.
            
    \section{Erträge in einem Unternehmen}
        \begin{itemize}
            \item Umsatzerlöse (Kerngeschäft)
            \item Kapital- und Zinserträge
            \item Gewinnbeteiligungen
            \item Dividenden
            \item Subventionen
            \item Mieterträge
            \item Verkauf von Gütern des Anlagevermögens
            \item ...
        \end{itemize}
        
    \section{Aufwendungen in einem Unternehmen}
        \begin{itemize}
            \item Materialien
            \item Gehälter
            \item Mieten
            \item Zinsen
            \item Steuern
            \item Werbeausgaben
            \item KFZ
            \item ...
        \end{itemize}
        
    \section{Ziele in einem Unternehmen}
        Grundsätzlich wird zwischen kurzfristigen Zielen mit einer Länge von nicht mehr als einem Jahr und langfristigen Zielen mit einer Länge von mehr als einem Jahr unterschieden.
        
        \subsection{Zielkategorie}
            \paragraph{Ökonomischen Ziele}
                Die ökonomischen Ziele werden von den Eigenkapitalgebern vertreten.
                
                \begin{itemize}
                    \item Gewinnmaximierung
                    \item Shareholder Value
                    \item Rentabilität
                    \item Unternehmenssicherung
                    \item Unternehmenswachstum
                \end{itemize}
                
            \paragraph{Soziale Ziele}
                Die sozialen Ziele werden von den Arbeitnehmern vertreten.

                \begin{itemize}
                    \item gerechte Entlohnung
                    \item gute Arbeitsbedingungen
                    \item betriebliche Sozialleitungen
                    \item Arbeitsplatzsicherheit
                    \item Mittbestimmung
                \end{itemize}
                
            \paragraph{Ökologische Ziele}
                Die ökologischen Ziele werden von der Öffentlichkeit vertreten.
            
                \begin{itemize}
                    \item Ressourcenschonung
                    \item Begrenzung von Schadstoffemission
                    \item Abfallvermeidung
                    \item Abfallrecycling
                \end{itemize}
            

        \subsection{Zielmerkmale}        
            \begin{tabular}{l l}
                \textbf{Zielmerkmal} & \textbf{Zielausprägung} \\
                \hline
                Zielsetzungsinstanz & individuell oder institutionell \\
                Zielinhalte & Mengengrößen, Geldgrößen, Sach- und Formalziele \\
                Zielausmaß & begrenzt oder unbegrenzt \\
                Zeitbezug & kurzfristig oder langfristig \\
                Zielbeziehung & komplementär, konkurrierend oder indifferent \\
                Rangordnung & Oberziele, Zwischenziele und Unterziele
            \end{tabular}
            
            \subsubsection{Sachziele und Formalziele}
                \paragraph{Sachziele}
                    Festlegung von
                    \begin{itemize}
                        \item Arten
                        \item Mengen
                        \item Qualitäten
                        \item Orten
                        \item Zeiten
                    \end{itemize}
                    der Produktion.
            
                \paragraph{Formalziele}
                    Festlegen von
                    \begin{itemize}
                        \item Umsatzzielen
                        \item Kostenzielen
                        \item Gewinnzielen
                        \item Rentabilitätszielen
                        \item Liquiditätszielen
                    \end{itemize}
                    
            \subsubsection{Zielbeziehungen}
                \begin{itemize}
                    \item Komplementäre Ziele unterstützen sich gegenseitig.
                    \item Konkurrierende Ziele behindern sich gegenseitig.
                    \item Indifferente Ziele sind unabhängig voneinander.
                \end{itemize}
                
        \subsection{Hierarchische Planung}
            % TODO: schreiben (https://moodle2.nordakademie.de/pluginfile.php/101037/mod_resource/content/1/1%20-%20ABWL%20Einführung.pdf)
            
        \subsection{Grundsätze der Zielsystembildung}
            % TODO: schreiben (https://moodle2.nordakademie.de/pluginfile.php/101037/mod_resource/content/1/1%20-%20ABWL%20Einführung.pdf)
        
    \section{Liquidität}
        \paragraph{Liquidemittel}
            \begin{itemize}
                \item Kassenbestände
                \item Bankguthaben
            \end{itemize}
            
        \paragraph{kurzfristige Verbindlichkeiten}
            \begin{itemize}
                \item Lieferanten
                \item Gehälter
                \item Mieten
                \item Kontokorrentkredit\footnote{Kredit, den die Bank automatisch beim Überziehen eines Girokontos gewehrt.}
                \item Steuern (Umsatz-, Körperschafts- und Gewerbesteuer)
            \end{itemize}
    
        \subsection{Liquidegrade}
            Es wird in drei Liquiditätsgrade unterschieden.
            
            \paragraph{Liquiditätsgrad 1}
                \begin{equation}
                    \frac{\text{Liquidemittel}}{\text{kurzfristige Verbindlichkeiten}} \cdot 100\% \geq 50\%
                \end{equation}
                
            \paragraph{Liquiditätsgrad 2}
                \begin{equation}
                    \frac{\text{Liquidemittel} + \text{Forderungen}}{\text{kurzfristige Verbindlichkeiten}} \cdot 100\% \geq 100\%
                \end{equation}
                
            \paragraph{Liquiditätsgrad 3}
                \begin{equation}
                    \frac{\text{Liquidemittel} + \text{Forderungen} + \text{Vorrat}}{\text{kurzfristige Verbindlichkeiten}} \cdot 100\% \geq 150\%
                \end{equation}        
    
    \section{Entscheidungsfindung}
        Entscheidungen können in verschiedene Kategorien unterteilt werden:
        
        \paragraph{Entscheidungen bei sicheren Erwartungen}
            \begin{itemize}
                \item[Konsequenzen] Bekannt
                \item[Eintrittswahrscheinlichkeiten] Bekannt
            \end{itemize}
        
        \paragraph{Entscheidung unter Risiko}
            \begin{itemize}
                \item[Konsequenzen] Unbekannt
                \item[Eintrittswahrscheinlichkeiten] Bekannt
            \end{itemize}
        
        \paragraph{Entscheidung bei unsicheren Erwartungen}
            \begin{itemize}
                \item[Konsequenzen] Unbekannt
                \item[Eintrittswahrscheinlichkeiten] Unbekannt
            \end{itemize}
            
        \subsection{Entscheidungen bei sicheren Erwartungen}
            \begin{tabular}{cc}
              Statisch & Dynamisch\\
              \hline
              Kostenvergleichsrechnung & Kapitalwertmethode \\
              Gewinnvergleichsrechnung & Interne-Zinsfuß-Methode \\
              Rentabilitätsrechnung & Dynamische Annuitätenrechnung \\
              Annuitätenrechnung &
            \end{tabular}

            \paragraph{Annuitätendarlehen}
                Bei einem Annuitätendarlehen gibt es einen jährlichen, festen Betrag. Dieser wird aufgeteilt in eine Zinskomponente und eine Tilgungskomponente.

            \paragraph{Barwert heute}
                Der Barwert heute sind abgezinsten Zahlungsströme.
                
            \subsubsection{Statische Verfahren}
                \paragraph{Kostenvergleichsrechnung}
                    Eine Kostenvergleichsrechnung einer neuen Produktionsanlage könnte folgendermaßen aussehen:
                    
                    \noindent + Kaufpreis\\
                    \noindent -- Kaufpreisminderungen (Skonto\footnote{Kaufpreisminderung aufgrund einer frühzeitigen Bezahlung}, Rabatt\footnote{Kaufpreisminderung direkt auf der Rechnung}, Subventionen\footnote{Unterstützung durch Organisationen oder den Staat}, Bonus\footnote{Rückzahlungen am Ende des Jahres.})\\
                    \noindent\rule{2cm}{0.4pt}\\
                    \noindent = Nettokaufpreis\\
                    \noindent + Kaufpreisnebenkosten (Fracht, Versicherung, Steuern, Zölle, Rollgeld)\\
                    \noindent\rule{2cm}{0.4pt}\\
                    \noindent = Nettokaufpreis inklusive Nebenkosten\\
                    \noindent + Montagekosten\\
                    \noindent + Fundament\\
                    \noindent + Schulungen  

                \paragraph{Gewinnvergleichsrechnung}
                    Bei der Gewinnvergleichsrechnung wird der Gewinn = Erlöse - Kosten verglichen:
                    
                    \noindent + Betriebskosten\\
                    \noindent + Wartungskosten\\
                    \noindent + Personalkosten\\
                    \noindent + Zinsaufwendungen\\
                    \noindent + Abschreibungen\\
                    \noindent\rule{2cm}{0.4pt}\\
                    \noindent = laufende Kosten\\                
                    \noindent\rule{8cm}{0.4pt}\\
                    \noindent + Verkaufserlöse\\
                    \noindent\rule{2cm}{0.4pt}\\
                    \noindent = Erlöse\\
                    
                    \subparagraph{Abschreibungen}
                        Abschreibungen sind Absetzungen für Abnutzung (AfA).
                        
                        \textbf{Beispiel:}\\
                        Nettokaufpreis inklusive Nebenkosten: 960,000\\
                        Nutzungsdauer: 8 Jahre
                        
                        Wenn eine lineare Abschreibung angewendet wird, können jedes Jahr 120,000 abgeschrieben werden. Da Abschreibungen den Gewinn reduzieren, muss das Unternehmen weniger Steuern zahlen.
                        
                \paragraph{Rentabilitätsrechnung}
                    Bei der Rentabilitätsrechnung wird die Rentabilität verglichen.
                    
                \paragraph{Annuitätenrechnung}
                    Bei der Annuitätenrechnung kumuliert man die Gewinne der Jahre und ermittelt so das Jahr, in dem sich die Anlage annulliert hat.
                    
            \subsubsection{Dynamische Verfahren}
                \paragraph{Kapitalwertmethode}
                    \begin{equation}
                        C_0 = \left(e - a\right)\frac{\left(1 + i\right)^n - 1}{i \cdot \left( 1 + i \right)^n} + R\cdot \left( 1 + i \right)^{-n} - A_0
                    \end{equation}
                    
                    \begin{itemize}
                        \item[$e$] Einzahlungen/Einnahmen
                        \item[$a$] Auszahlungen/Ausgaben
                        \item[$i$] Renditeerwartungen des Managements
                        \item[$n$] Nutzungsdauer
                        \item[$R$] Restwert
                        \item[$A_0$] Anschaffungsauszahlungen
                    \end{itemize}
                    
                    \subparagraph{Gewinn}
                        Der Gewinn wird mit $\left(e - a\right)$ in der Formel ausgedrückt.
                    
                    \subparagraph{Abzinsfaktor}
                        Der Abzinsfaktor (AbF) $\left( 1 + i \right)^{-n}$ wird verwendet, um einen Betrag über $n$ Jahre abzuzinsen. Das Bedeutet, wenn man nach $n$ Jahren einen Restbetrag von $x$ hat, hätte man auch $x$ für $n$ Jahre bei einem Zinssatz von $i$ anlegen können.
                        
                    \subparagraph{Aufzinsfaktor}
                        Der Aufzinsfaktor (AuF) $\left( 1 + i \right)^{n} = \frac{1}{\text{AbF}}$ wird verwendet, um einen Betrag über $n$ Jahre aufzuzinsen.
                        
                    \subparagraph{Diskontierungsfaktor}
                        Der Diskontierungsfaktor (DSF) $\frac{\left(1 + i\right)^n - 1}{i \cdot \left( 1 + i \right)^n}$ wird verwendet, um einen jährlichen Gewinn über $n$ Jahre abzuzinsen.
                        
                    \subparagraph{Kapitalwiedergewinnungsfaktor}
                        Der Kapitalwiedergewinnungsfaktor (KwF) ist das Reziprok des Diskontierungsfaktor.
                        
                    \subparagraph{Bedeutung von $C_0$}
                        Wenn man mit der Kapitalwertmethode $C_0 = 0$ berechnet, erfüllt das Vorhaben die Renditeerwartung $i$ genau. Bei Werten über $0$ hat man eine höhere Rendite. Bei Werten kleiner als $0$ muss an den Parametern gearbeitet oder das Vorhaben aufgegeben werden.
                        
                \paragraph{Gestaffelte Kapitalwertmethode}
                    In vielen Fällen ist der Gewinn über die Jahre nicht konstant sondern ändert sich. In diesem Fall muss man die gestaffelte Kapitalwertmethode anwenden.
                    
                    Bei der gestaffelten Kapitalwertmethode wird zunächst der Zeitwert der einzelnen Staffeln berechnet. Dafür wird der Gewinn mit dem Diskontierungsfaktor der Dauer der Staffel multipliziert. Anschließend wird die Summe der der Zeitwert multipliziert mit dem Abzinsfaktor berechnet. Gegebenenfalls muss noch der Restwert über die gesamte Zeit berechnet werden.
                    
                    Bei I Staffeln ohne Restwert könnte die Formal folgendermaßen aussehen
                    
                    \begin{equation}
                        \sum_{i=1}^I \left(\text{Gewinn d. Staffel $i$}\right) \cdot \left(\text{DSF d. Dauer d. Staffel $i$}\right) \cdot \left(\text{AbF d. Zeit bis zur Staffel $i$}\right) - A_0
                    \end{equation}
                    
            \subsection{Nutzwertanalyse}
                Bei der Nutzwertanalyse werden zuvor festgelegte Kriterien gewichtet und anschließend mit einer Scoring Tabelle bewertet. Der Gesamtscore ist die Summe der Produkte der Faktoren und Einzelscores:
                
                \begin{equation}
                    \text{Score} = \sum_{i=1}^N \text{Faktor $i$} \cdot \text{Score $i$}
                \end{equation}
                        
        
\end{document}
